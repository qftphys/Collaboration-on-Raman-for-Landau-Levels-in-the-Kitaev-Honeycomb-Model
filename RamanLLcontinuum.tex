%% The layout here is based on the APS REVTeX 4 template, Version 4.1r of REVTeX, August 2010.

% Group addresses by affiliation; use superscriptaddress for long
% author lists, or if there are many overlapping affiliations.
% For Phys. Rev. appearance, change preprint to twocolumn.
% Choose pra, prb, prc, prd, pre, prl, prstab, prstper, or rmp for journal
%  Add 'draft' option to mark overfull boxes with black boxes
%  Add 'showpacs' option to make PACS codes appear
%  Add 'showkeys' option to make keywords appear

%latexdiff-so file1.tex file2.tex > diff.tex

\documentclass[reprint,amsmath,amssymb,aps,prb,groupedaddress,nofootinbib,superscriptaddress]{revtex4-1}
%\documentclass[aps,prl,preprint,superscriptaddress]{revtex4-1}
%\documentclass[aps,prl,reprint,groupedaddress]{revtex4-1}

\usepackage{amsthm,amssymb,amsmath,braket,mathdots}
\usepackage{bm}
\usepackage[hidelinks,pagebackref=false,pdfnewwindow=true]{hyperref} %\hypersetup{draft}
%\usepackage{doi}
\usepackage{epstopdf,psfrag}
\usepackage{relsize,amsbsy}
\usepackage[export]{adjustbox}

\usepackage{graphicx,xcolor,tikz}


\DeclareMathOperator{\sgn}{sgn}
\DeclareMathOperator{\tr}{Tr}
\DeclareMathOperator{\re}{Re}
\DeclareMathOperator{\im}{Im}

\newcommand{\dd}{\partial}
\newcommand{\1}{\mathds{1}}

\newcommand{\mac}{\mathcal}
\newcommand{\be}{\begin{equation}}
\newcommand{\ee}{\end{equation}}

%\newcommand{\dd}{\partial}
%\newcommand{\1}{\mathds{1}}
\newcommand{\ms}{Majorana spinon }
\newcommand{\mss}{Majorana spinons }
\newcommand{\mssns}{Majorana spinons}
\newcommand{\hh}{harmonic honeycomb }
\newcommand{\HH}{Harmonic Honeycomb }
\newcommand{\Hs}[1]{\mbox{$\mathcal{H}$-$#1$}}

\newcommand{\bit}{\begin{enumerate}}
	\newcommand{\eit}{\end{enumerate}}
\newcommand{\m}{\item}
\newcommand{\zt}{\mathbb{Z}_2}
\def\em{\it}
\definecolor{bananayellow}{rgb}{1.0, 0.88, 0.21}
\definecolor{straw}{rgb}{0.32, 0.28, 0.1}

\newcommand{\fb}{\it \color{orange}}
\definecolor{palatinatepurple}{rgb}{0.49, 0.24, 0.46}
\newcommand{\bp}{ \color{straw} }
\definecolor{darkblue}{rgb}{0.0, 0.0, 0.55}
\newcommand{\np}{ \color{darkblue}}
\definecolor{darkgreen}{rgb}{0.0, 0.5, 0.0}
\newcommand{\jk}{\it \color{darkgreen}}

\graphicspath{{Figures/}}

\begin{document}
	
	
	
	\title{Majorana Landau Level Spectroscopy  --  a proposal for observing pseudo magnetic fields in  strained thin films of $\alpha$-RuCl$_3$}
		\author{us}
%		\affiliation{\small School of Physics and Astronomy, University of Minnesota, Minneapolis, Minnesota 55455, USA}
%		%	\email[]{perre035@umn.edu}
%		%	%\homepage[]{https://www.physics.umn.edu/people/perreault.html}
%		%	
%		\author{Johannes Knolle}
%		\affiliation{\small Department of Physics, Cavendish Laboratory, JJ Thomson Avenue, Cambridge CB3 0HE, U.K.}
%		%	
%		\author{Natalia B. Perkins}
%		%	
%		\author{F. J. Burnell}
%		\affiliation{\small School of Physics and Astronomy, University of Minnesota, Minneapolis, Minnesota 55455, USA}
		
		\date{\today} 
		
		
		\begin{abstract}
		 	abstract
		\end{abstract}
		
		
		% insert suggested PACS numbers in braces on next line
		%\pacs{}
		% insert suggested keywords - APS authors don't need to do this
		%\keywords{}
		
		%\maketitle must follow title, authors, abstract, \pacs, and \keywords
		\maketitle
		
		%	\tableofcontents
		
		
		%{\np Natasha }	
		%{\jk Johannes}
		%{\bp Brent} {\fb Fiona}
	
	%\tableofcontents
	
	\section{Introduction}
	
Lattice strain can induce effective magnetic fields without actually breaking time reversal symmetry. An out of plane (along the $z$-axis) magnetic field can be induced by strain fields of the form $B=\beta \left[2 \partial_x u_{xy}-\partial_y (u_{xx}-u_{yy}) \right]$. Here, in addition to our numerical evaluations we try to obtain a more microscopic understanding of the influence of such a pseudo magnetic field on a Majorana fermion system and how the expected Landau level degeneracy is manifest in the Raman response function. 

Via minimal coupling of the vector potential we use the canonical momenta $\mathbf{\Pi}= \mathbf{p}-\frac{e}{c} \mathbf{A}$ and work in the Landau gauge $\mathbf{A}=B(0,x)$. We study the low energy behaviour of our Majorana fermion systems in the flux free low temperature sector by expanding the Majorana field operators $\hat \Psi = \left( \Psi_{A}(\mathbf{r}) \Psi_{B}(\mathbf{r}) \right)^T$ around the two Dirac points (labeled by $\nu=\pm 1$)
\begin{eqnarray}
H_{\nu} \approx i \int \text{d}^2 \mathbf{r} 
\hat \Psi^{\dagger}
\begin{pmatrix}
0 & \nu p_x- i(p_y- \nu B x) \\
- \nu p_x- i(p_y-\nu B x) & 0 
\end{pmatrix}
\hat \Psi \nonumber .
\end{eqnarray}
Note that $B$ has opposite sign at the two Dirac points leaving TRS intact, which however prevents the usual trick of combining the two Majorana cones into a single cone of complex fermions. Nevertheless, we can concentrate on only one of the cones because there is no coupling between the two.  


We introduce the ladder operator $a=\frac{l}{2\hbar} \left(\Pi_x-i \Pi_y \right)$ with $l^2=\frac{c\hbar}{e|B|}$. Then we expand the field operators in terms of the standard Landau Level wave functions $\Psi_{A} (\mathbf{r}) = \sum_{n,p} \Phi_{n-1,p} (\mathbf{r}) f_{A n p}$ and $\Psi_{B} (\mathbf{r}) = \sum_{n,p} \Phi_{n,p} (\mathbf{r}) f_{B n p}$ with Majorana operators $f_{A/B n p}$ obeying the Majorana commutation relations (but note the sign change of momenta from conjugation $\Psi_{B}^{\dagger} (\mathbf{r}) = \sum_{n,p} \Phi^*_{n,p} (\mathbf{r}) f_{B n -p}$). Using the ladder operator properties $a \Phi_n=\sqrt{n} \Phi_{n-1}$ and $a^{\dagger} \Phi_n=\sqrt{n+1} \Phi_{n+1}$ we obtain the Hamiltonain
\begin{eqnarray}
H_{+}  & \approx & 
\begin{pmatrix}
f_{An-p} & f_{B n -p} 
\end{pmatrix}
\begin{pmatrix}
0 & i \frac{2 \hbar}{l} \sqrt{n} \\
- i \frac{2 \hbar}{l} \sqrt{n} & 0 
\end{pmatrix}
\begin{pmatrix}
f_{A n p} \\ f_{B n p} 
\end{pmatrix}.
\end{eqnarray}
It is diagonalized by introducing  the complex fermions $a_{np}$ with $f_{Bnp}=a_{np}+a^{\dagger}_{n-p}$ and $f_{Anp}=i \left(a_{np}  -a_{n-p}^{\dagger}\right)$ such that
\begin{eqnarray}
H_{+}  & \approx & 
\sum_{np} E(n) \left[ a_{np}^{\dagger} a_{np} -\frac{1}{2}\right] 
\end{eqnarray}
with the energies $E(n)=4 \frac{\sqrt{2\hbar}}{l} \sqrt{n}$ obeying the well known $\propto\sqrt{n}$ scaling of Dirac fermions. 

Next, we look at the low energy behaviour of the Raman response $I(\omega)=\int_{-\infty}^{\infty} \text{d}t e^{i\omega t} \langle R(t) R(0)\rangle$. The main difference between the non-resonant and resonant Raman vertices is that the former couples both sublattices whereas the latter does not
\begin{eqnarray}
R_{\text{non-res}} & \propto & i \int \text{d}^2 \mathbf{r}  \Psi^{\dagger}_A(\mathbf{r})(t) \Psi_B(\mathbf{r}) \\ \nonumber
& \propto & \sum_{np} \left[ a_{n-p}(t) -a^{\dagger}_{np} (t) \right] \left[ a_{n-1 p} + a^{\dagger}_{n-1 -p} \right] \\ \nonumber
R_{\text{res}} & \propto & i \int \text{d}^2 \mathbf{r} \left[ \Psi^{\dagger}_A(\mathbf{r})(t) \Psi_B(\mathbf{r+\delta}) - \Psi^{\dagger}_A(\mathbf{r})(t) \Psi_B(\mathbf{r-\delta}) \right] \\ \nonumber
& \propto & \sum_{np} \sin \left(p \delta \right) \left[ a_{n-p}(t) -a^{\dagger}_{np} (t) \right] \left[ a_{n p} + a^{\dagger}_{n -p} \right] .
\end{eqnarray}
The form of both vertices is very similar but only the non-resonant vertex mixes states which differ by one LL index. From the time dependence $a_{np}(t)=a_{np} e^{-i t E(n)}$ and $a_{np} |0\rangle=0$ we can directly calculate the low energy Raman responses
\begin{eqnarray}
I_{\text{non-res}} & \propto &  \sum_{np} \delta\left[\omega - E(n)-E(n+1) \right] \\ \nonumber
I_{\text{res}} & \propto &  \sum_{np} \delta\left[\omega - 2E(n)  \right] .
\end{eqnarray}
In agreement with the numerical observation we find different scaling of the resonant and non-resonant intensities which originates from the sub-lattice selectivity of the vertices and the special structure of LL wave functions in Dirac systems which mix adjacent LL indices. 

\section{Finite Temperature Formalism} 
By the simple form of the canonical ensemble we can write 
\begin{align}\label{iflux}
I(\omega) \propto \sum_{\text{ flux sectors } M} e^{-\beta E_0^M} I^M(\omega)
\end{align}
where $E_0^M$ is the energy of the lowest-energy state in the flux-configuration $M$. Within each flux sector the Hamiltonian can be written in the Kitaev fermionization %$\sigma^\alpha_j = i b^\alpha_j c_j$ 
as
\begin{align}
\mathcal{H} =  \frac{1}{2}\sum_{\left<rr'\right>} J^{\alpha} u^\alpha_{\left<rr'\right>} i c_r c_{r'} \equiv  \frac{1}{2}\sum_{r,r'} H_{rr'} c_r c_{r'},
\end{align}
where $u^\alpha_{\left<rr'\right>} = i b^\alpha_r b^\alpha_{r'} = \pm 1$. 
%{\it Diagonalization.} 
In terms of these Majorans $H$ has a chiral symmetry $S$ that flips sign of $c_r$ on one of two sublattices (so that $\{S,H\} = 0$). % This is common for the Kitaev model on a bipartite lattice that can lead to a sublattice symmetry within the unit cell. 
In the basis where $S$ is diagonal $H$ takes the block-off-diagonal form
\begin{align}
H = i\left(\begin{array}{cc}
0 & G \\
-G^\dag & 0
\end{array}\right).
\end{align}
In this case the a diagonalization of $H$ can be obtained from the singular value decomposition of $G$. Given unitary $u$ and $v$ %$n/2 \times n/2$ 
such that \mbox{$\ u^\dagger G v = \epsilon/4$}, then  %, where again all of the eigenvalues are positive. % [One algorithm is to find $u_k$ as the unitary matrix diagonalizing $G_k G^\dagger_k$, whose eigenvalues are $(\epsilon_k^\mu)^2$. Then $v_k$ comes from matrix inversion.] 
%The particle-hole redundancy forces $U_k$ to take the following form reflecting the fact that are only $n/2$ fermionic quasiparticles.
\begin{align}\label{U}
U = \frac{1}{\sqrt{2}}\left(\begin{array}{cc}
u & u \\
-iv & iv
\end{array}\right).
\end{align}
Now
%Let $U^{\mu\nu}$ be the unitary matrix that diagonalizes the Hermitian matrix $H^{\mu\nu}$. Therefore 
\begin{align}
U^\dag H U = \Omega = \left[\begin{array}{cc}
\text{diag}(\vec{\epsilon}) & 0 \\
0 & -\text{diag}(\vec{\epsilon})
\end{array}\right],
\end{align}
with $\epsilon^\mu \geq 0$.
%where the positive and negative eigenvalue pairing comes from the particle-hole redundancy of the Nambu-Bogoliubov-deGennes form, as shown in Ref. \onlinecite{Blaizot86}. 
%The case of usual fermions treated in that reference can be related to the Majorana description by the standard pairing $$\left(\begin{array}{c}f_k \\ f_{-k}^\dag 
%\end{array}\right) = \left(\begin{array}{cc}1 & i \\ 1& -i 
%\end{array}\right) \left(\begin{array}{c}c^A_k \\ c^B_k 
%\end{array}\right).$$
We can define operators $a_\lambda = U^\dag_{\lambda \lambda'} c_\lambda$ for $\lambda = 1,...,n/2$ to get the set of $n/2$ fermionic quasiparticles \mbox{$\{a^\dagger_\lambda , a_{\lambda'} \}= \delta_{\lambda,\lambda'}$} so the Hamiltonian becomes
\begin{align}\label{Ha}
\mathcal{H} =  \frac{1}{2}\sum_{\lambda}\epsilon^\mu \left[2 a_\lambda^\dagger a_\lambda - 1 \right].
\end{align}
Therefore the excitation created by $(a^\dagger)^\mu$ has energy $\epsilon^\mu$.

%where $u_k$ and $v_k$ are themselves Unitary matrices.
%We can now define $n$ new Majorana fermions
%\begin{align}
%\gamma^{A,\mu}_k &= (u_k^\dag)^{\mu \nu} c^{A,\mu}_k \nonumber \\
%\gamma^{B,\mu}_k &= (v_k^\dag)^{\mu \nu} c^{B,\mu}_k,
%\end{align}
%where we have split up the vector $\vec{c}_k = ( \vec{c}_k^A , \vec{c}_k^B )$ into two halves. These form Majorana fermions $\{\gamma^{A,\mu}_{-k},\gamma^{A,\nu}_k\}= \delta^{\mu,\nu}$ (and so on). Complex fermionic quasiparticles can be formed by $a_k^\mu = \frac{1}{\sqrt{2}}(\gamma^{A,\mu} + i\gamma^{B,\mu})$ so that
We can view the quasiparticles formed by $u$ and $v$ as $n$ Majorana fermions, while the formation of $U$ from $u$ and $v$ corresponds to the usual Dirac-fermionization of a pair of Majoranas.
%\begin{align}
%\gamma^{A,\mu}_k &= (u_k^\dag)^{\mu \nu} c^{A,\mu}_k \nonumber \\
%\gamma^{B,\mu}_k &= (v_k^\dag)^{\mu \nu} c^{B,\mu}_k,
%\end{align}
%where we have split up the vector $\vec{c}_k = ( \vec{c}_k^A , \vec{c}_k^B )$ into two halves. These can be trivially diagonalized in the form of fermionic quasiparticles by $a_k^\mu = \frac{1}{\sqrt{2}}(\gamma^{A,\mu} + i\gamma^{B,\mu})$ These give the same relation as Eq.~(\ref{Ha}) with
%\begin{align}
%U_k = \frac{i}{\sqrt{2}}\left(\begin{array}{cc}
%u_k & u_k \\
%-iv_k & iv_k
%\end{array}\right).
%\end{align}

Following 1602.05277 the Raman operator for the Kitaev model is given by 
\begin{align}
\mathcal{R} &= \sum_{\alpha = x,y,z} \sum_{\left<ij\right>_\alpha} \left({\boldsymbol \epsilon}_{\rm in} \cdot \mathbf{d}^\alpha \right) \left({\boldsymbol \epsilon}_{\rm out} \cdot \mathbf{d}^\alpha \right) J^\alpha S^\alpha S^\alpha  \\
& = \sum_{\left<rr'\right>} \left({\boldsymbol \epsilon}_{\rm in} \cdot \mathbf{d}^\alpha \right) \left({\boldsymbol \epsilon}_{\rm out} \cdot \mathbf{d}^\alpha \right) H_{rr'} c_r c_{r'} \\ % & = \sum_{r,r'} B_{rr'} c_r c_{r'} \nonumber \\
&= \frac{1}{2}
\left(\begin{array}{c}
\mathbf{c}_A \\
\mathbf{c}_B 
\end{array}\right)^T
i\left(\begin{array}{cc}
A & B \\ -B^\dagger & A' 
\end{array}\right) \left(\begin{array}{c}
\mathbf{c}_A \\
\mathbf{c}_B 
\end{array}\right) \\
&= \frac{1}{2}
\left(\begin{array}{c}
\mathbf{a} \\
(\mathbf{a}^\dag)^T 
\end{array}\right)^\dagger
\left(\begin{array}{cc}
C & D \\ D^\dagger & -C 
\end{array}\right) \left(\begin{array}{c}
\mathbf{a}_{\lambda'} \\
(\mathbf{a}_{\lambda'}^\dag)^T 
\end{array}\right).
\end{align}
Here \mbox{$C = u^\dag B v + v^\dag B^\dag u$} and \mbox{$D = -u^\dag B v + v^\dag B^\dag u$} for a Raman operator that is symmetric w.r.t. swapping in and out polarizations and \mbox{$C = u^\dag A u + v^\dag A' v$} and \mbox{$D = u^\dag A u - v^\dag A' v$} for an antisymmetric channel. % where $B_{rr'}= \left({\boldsymbol \epsilon}_{\rm in} \cdot \mathbf{d}^\alpha \right) \left({\boldsymbol \epsilon}_{\rm out} \cdot \mathbf{d}^\alpha \right) G_{rr'}$ is the non-zero part of the Raman operator. 
Then finally,
\begin{align}
I^M(\omega) \propto& \sum_{\lambda \lambda'} \left[ 2 |C_{\lambda \lambda'}|^2 f(\varepsilon_\lambda)[1-f(\varepsilon_{\lambda'})] \delta(\omega+\varepsilon_\lambda -\varepsilon_{\lambda'}) \right. \nonumber \\
&\left. +|D_{\lambda \lambda'}|^2 [1-f(\varepsilon_{\lambda})][1-f(\varepsilon_{\lambda'})] \delta(\omega-\varepsilon_\lambda -\varepsilon_{\lambda'})
\right] ,
\end{align}
where $B$ and therefore $C$ and $D$ depend on the gauge chosen for each flux configuration $M$.

We evaluate the sum (\ref{iflux}) using a classical Monte-Carlo approach following the VEGAS algorithm where we sample the entire two-particle spinon spectrum of pseudorandomly-chosen flux configurations until the expected error in the total $I(\omega)$ is sufficiently low. This is made slightly faster by treating the summand as a different random function for each flux sector of $p$ total fluxes, independent of their placement. %This is made slightly easier since the spectra of flux sectors with the same number of fluxes $n$ turn out to be similar to each other when compared to other sectors . 







\section*{Acknowledgements}
We acknowledge helpful discussions with S. Rachel. The work of J.K. is supported by a Fellowship within the Postdoc-Program of the German Academic Exchange Service (DAAD). NP acknowledges the support from NSF DMR-1511768. FJB is supported by NSF DMR-1352271 and Sloan FG-2015- 65927.


	
	%\bibliographystyle{apsrev}
	%\bibliography{References}
	
\end{document}